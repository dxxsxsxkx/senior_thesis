% created manually
\begin{table}
\begin{threeparttable}
\begin{tabular}{l|cccccccc}
\toprule
Bloc & 1996 & 2000 & 2003 & 2005 & 2009 & 2012 & 2014 & 2017 \\
\midrule
Hokkaido & 9 & 8 & 8 & 8 & 8 & 8 & 8 & 8 \\
Tohoku & 16 & 14 & 14 & 14 & 14 & 14 & 14 & 13 \\
Kita-kanto & 21 & 20 & 20 & 20 & 20 & 20 & 20 & 19 \\
Tokyo & 19 & 17 & 17 & 17 & 17 & 17 & 17 & 17 \\
Minami-kanto & 23 & 21 & 22 & 22 & 22 & 22 & 22 & 22 \\
Hokuriku Shinetsu & 13 & 11 & 11 & 11 & 11 & 11 & 11 & 11 \\
Tokai & 23 & 21 & 21 & 21 & 21 & 21 & 21 & 21 \\
Kinki & 33 & 30 & 30 & 30 & 29 & 29 & 29 & 28 \\
Chugoku & 13 & 11 & 11 & 11 & 11 & 11 & 11 & 11 \\
Shikoku & 7 & 6 & 6 & 6 & 6 & 6 & 6 & 6 \\
Kyushu & 23 & 21 & 21 & 21 & 21 & 21 & 21 & 20 \\
\bottomrule
\end{tabular}
\begin{tablenotes}[flushleft]
  \scriptsize{
    \item Magnitudes of each PR regional district, 1996 - 2017. 
    \item \textit{Data source}: Reed and Smith (2017); Ministry of Internal Affairs and Communications(2017).
  }
\end{tablenotes}
\end{threeparttable}
\caption{Magnitudes of PR Districts}
\label{table:distM}
\end{table}